{
  \vspace{0.5em}
  \begin{center}
    \refstepcounter{table}
    Table \thetable: Instruction Fetch Module interface signals\label{tab:ifm-interface}
  \end{center}

\footnotesize
\begin{xltabular}{0.9\textwidth}{|l|c|c|X|}
  \hline
  \cellcolor{gray!20}\textbf{NAME} & \cellcolor{gray!20}\textbf{TYPE} & \cellcolor{gray!20}\textbf{WIDTH} & \cellcolor{gray!20}\textbf{DESCRIPTION} \\
  \hline
  clk\_i & I & 1 & Clock input. \\
  \hline
  rst\_i & I & 1 & Reset input. \\
  \hline
  \multicolumn{4}{|l|}{\textbf{JUMP LOGIC}} \\
  \hline
  irq\_i & I & 1 & External interrupt request. \\
  \hline
  drq\_i & I & 1 & External debug request. \\
  \hline
  branch\_i & I & 1 & Branch request. \\
  \hline
  boffset\_i & I & 20 & Branch offset from the pc of the instruction in the execute stage (pc - 8).  \\
  \hline
  \multicolumn{4}{|l|}{\textbf{WISHBONE MASTER}} \\
  \hline
  wb\_adr\_o & O & 32 & The address output array is used to pass binary address. \\
  \hline
  wb\_dat\_i & I & 32 & The data input array is used to pass binary data. \\
  \hline
  wb\_we\_o & O & 1 & The write enable output indicates whether the current local bus cycle is a READ or WRITE cycle. This signal is negated during READ cycles and is asserted during WRITE cycles. This signal is always low as this interface only supports READ cycles. \\
  \hline
  wb\_sel\_o & O & 4 & The select output array indicates where valid data is expected on the wb dat i signal array during READ cycles. Each individual select signal correlates to one of four active bytes on the 32-bit data port. \\
  \hline
  wb\_stb\_o & O & 1 & The strobe output indicates a valid data transfer cycle. It is used to qualify various other signals on the interface. \\
  \hline
  wb\_ack\_i & I & 1 & The acknowledge input, when asserted, indicates the nor- mal termination of a bus cycle. \\
  \hline
  wb\_cyc\_o & O & 1 & The cycle output, when asserted, indicates that a valid bus cycle is in progress. This signal is asserted for the duration of all bus cycles. \\
  \hline
  wb\_stall\_i & I & 1 & The pipeline stall input indicates that current slave is not able to accept the transfer in the transaction queue. \\
  \hline
  \multicolumn{4}{|l|}{\textbf{OUTPUT LOGIC}} \\
  \hline
  output\_ready\_i & I & 1 & Output handshaking signal asserted when the destination is ready to receive the output \\
  \hline
  output\_valid\_o & O & 1 & Output handshaking signal asserted when the output is valid. \\
  \hline
  instr\_o & O & 32 & Instruction output. \\
  \hline
\end{xltabular}
}
