\section{Introduction}
  \subsection{Purpose}

    \begin{content}
        This documents aims at defining the requirements for ECAP5-DPROC as well as describing its architecture. Both user and product requirements will be covered.
      \end{content}

  \subsection{Intended Audience and Use}

    \begin{content}
        This document targets hardware engineers who shall implement ECAP5-DPROC by refering to the described architecture. It is also intended for system engineers working on the integration of ECAP5-DPROC in ECAP5. Finally, this document shall be used as a technical reference by software engineers configuring ECAP5-DPROC through hardware-software interfaces.
      \end{content}

  \subsection{Product Scope}

    \begin{content}
        ECAP5-DPROC is an implementation of the RISC-V instruction set architecture targetting \textit{Educational Computer Architecture Platform 5} (ECAP5). It will provide the main means of software execution in ECAP5.
      \end{content}

  \subsection{Conventions}

    Requirements shall be described here.

    Requirement relationships :
    \begin{itemize}
        \item Composition
        \item Derivation
        \item Refinement
        \item Satisfy
        \item Verify
        \item Copy
      \end{itemize}

    \begin{content}
      Events provide a formal identifier for describing the design of hardware systems. They can be triggered by requirements and requirements can check if an event is triggered. A sample event is represented in figure \ref{fig:event-sample}.
      \end{content}

    \begin{figure}[H]
        \centering
        {
          \footnotesize
          \begin{tabularx}{\eventtablelength}{|p{2.5cm}|X|}
            \hline
            \cellcolor{\eventcolor!30}\textbf{Event ID} & Formal event identifier \\
            \hline
            \cellcolor{\eventcolor!15}\textbf{Description} & \cellcolor{\eventcolor!5}Informal event description \\
            \hline
          \end{tabularx}
        }
        \caption{Representation of a sample event}
        \label{fig:event-sample}
      \end{figure}

    The bit indexing shall be described somewhere.

    Byte size as well.

    Inputs missing from timing diagrams are considered low or undefined.

    Italic names are timing diagram parameters.

  \subsection{Definitions and Abbreviations}

    hardware-configurable

    software-configurable

  \subsection{References}

    \begin{center}
      {
        \vspace{0.5em}
        \small
        \begin{tabularx}{0.9\textwidth}{|c|c|X|}
            \hline
            \cellcolor{gray!20}\textbf{Date} & \cellcolor{gray!20}\textbf{Version} & \cellcolor{gray!20}\textbf{Title} \\
            \hline
            December 13, 2019 & 20191213 & The RISC-V Instruction Set Manual Volume I: User-Level ISA \\
            \hline
            March 22, 2019 & 0.13.2 & RISC-V External Debug Support \\
            \hline
            June 22, 2010 & B.4 & WISHBONE System-on-Chip (SoC) Interconnection Architecture for Portable IP Cores \\
            \hline
          \end{tabularx}
        \vspace{0.5em}
      }
      \end{center}

\newpage
