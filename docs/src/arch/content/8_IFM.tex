\section{Instruction Fetch Module}

  \subsection{Interface}

    \begin{content}
        The instruction fetch module handles fetching from memory the instructions to be executing. The signals are described in table \ref{tab:ifm-interface}. 
      \end{content}

    {
  \vspace{0.5em}
  \begin{center}
    \refstepcounter{table}
    Table \thetable: Instruction Fetch Module interface signals\label{tab:ifm-interface}
  \end{center}

\footnotesize
\begin{xltabular}{0.9\textwidth}{|l|c|c|X|}
  \hline
  \cellcolor{gray!20}\textbf{NAME} & \cellcolor{gray!20}\textbf{TYPE} & \cellcolor{gray!20}\textbf{WIDTH} & \cellcolor{gray!20}\textbf{DESCRIPTION} \\
  \hline
  clk\_i & I & 1 & Clock input. \\
  \hline
  rst\_i & I & 1 & Reset input. \\
  \hline
  \multicolumn{4}{|l|}{\textbf{JUMP LOGIC}} \\
  \hline
  irq\_i & I & 1 & External interrupt request. \\
  \hline
  drq\_i & I & 1 & External debug request. \\
  \hline
  branch\_i & I & 1 & Branch request. \\
  \hline
  boffset\_i & I & 20 & Branch offset from the pc of the instruction in the execute stage (pc - 8).  \\
  \hline
  \multicolumn{4}{|l|}{\textbf{WISHBONE MASTER}} \\
  \hline
  wb\_adr\_o & O & 32 & Wishbone read address.  \\
  \hline
  wb\_dat\_i & I & 32 & Wishbone read data. \\
  \hline
  wb\_sel\_o & O & 4 & Wishbone byte selector. \\
  \hline
  wb\_stb\_o & O & 1 & Wishbone handshaking signal asserted when emitting a request. \\
  \hline
  wb\_ack\_i & I & 1 & Acknowledge. Indicates a normal termination of a bus cycle. \\
  \hline
  \multicolumn{4}{|l|}{\textbf{OUTPUT LOGIC}} \\
  \hline
  output\_ready\_i & I & 1 & Output handshaking signal asserted when the destination is ready to receive the output \\
  \hline
  output\_valid\_o & O & 1 & Output handshaking signal asserted when the output is valid. \\
  \hline
  instr\_o & O & 32 & Instruction output. \\
  \hline
\end{xltabular}
}


  \subsection{Specification}

    \subsubsection{PC register}

      \req{D\_IFM\_PC\_REGISTER\_01}{
          The instruction fetch module shall implement a \texttt{pc} register which shall store the address of the instruction to be fetched.
        }[
          derivedfrom=F\_REGISTERS\_03
        ]

      \req{D\_IFM\_PC\_INCREMENT\_01}{
          The value of the \texttt{pc} register shall be incremented by 4 on the rising edge of \texttt{clk\_i} after both \texttt{output\_ready\_i} and \texttt{output\_valid\_o} where asserted.
        }[
          derivedfrom=F\_REGISTERS\_03,
          rationale=Instructions are stored consecutively and aligned on a four-byte boundary.
        ]

      \req{D\_IFM\_PC\_INCREMENT\_02}{
        The value of the \texttt{pc} register shall be incremented by the value of \texttt{boffset\_i} on the rising edge of \texttt{clk\_i} when \texttt{branch\_i} is asserted. \texttt{pc} shall not be updated on subsequent asserted \texttt{branch\_i} until the pending memory requests are terminated.
        }[
          derivedfrom={F\_JAL\_02, F\_JALR\_02, F\_BEQ\_02, F\_BNE\_02, F\_BLT\_02, F\_BGE\_02, F\_BLTU\_02, F\_BGEU\_02}
        ]

      \req{D\_IFM\_PC\_LOAD\_01}{
          The value of the \texttt{pc} register shall be loaded with \texttt{IRQ\_HANDLER\_ADDR} on the rising edge of \texttt{clk\_i} when \texttt{irq\_i} is asserted.
        }[
          derivedfrom=F\_IRQ\_HANDLER\_01
        ]

      \req{D\_IFM\_PC\_LOAD\_02}{
          The value of the \texttt{pc} register shall be loaded with \texttt{DRQ\_HANDLER\_ADDR} on the rising edge of \texttt{clk\_i} when \texttt{drq\_i} is asserted.
        }[
          derivedfrom=F\_DRQ\_HANDLER\_01
        ]

      \req{D\_IFM\_PC\_REGISTER\_02}{
          Any changes to the \texttt{pc} register shall be performed before any memory requests.
        }

    \subsubsection{Reset}

      \req{D\_IFM\_RESET\_01}{
          The value of the \texttt{pc} register shall be loaded with \texttt{BOOT\_ADDR} on the rising edge of \texttt{clk\_i} following the assertion of \texttt{rst\_i}.
        }[
          derivedfrom=F\_REGISTERS\_RESET\_01
        ]

      \req{D\_IFM\_RESET\_02}{
          The \texttt{output\_valid\_o} signal shall be deasserted on the rising edge of \texttt{clk\_i} following the assertion of \texttt{rst\_i}.
        }

    \subsubsection{Fetch cancellation}
      
      \event{EV\_IFM\_REQUEST\_CANCEL\_01}{
          This event represents the cancellation of any pending memory requests
        }

      \req{D\_IFM\_REQUEST\_CANCEL\_01}{
          When \eventref{EV\_IFM\_REQUEST\_CANCEL\_01} is triggered, the memory interface shall wait for the termination of any pending memory requests before performing new requests.
        }

      \req{D\_IFM\_REQUEST\_CANCEL\_02}{
          On the rising edge of \texttt{clk\_i} following the assertion of \texttt{drq\_i}, \eventref{EV\_IFM\_REQUEST\_CANCEL\_01} shall be triggered.
        }[
          derivedfrom=I\_DRQ\_01
        ]

      \req{D\_IFM\_REQUEST\_CANCEL\_03}{
          On the rising edge of \texttt{clk\_i} following the assertion of \texttt{irq\_i}, \eventref{EV\_IFM\_REQUEST\_CANCEL\_01} shall be triggered.
        }[
          derivedfrom=I\_IRQ\_01
        ]

      \req{D\_IFM\_REQUEST\_CANCEL\_04}{
          On the rising edge of \texttt{clk\_i} following the assertion of \texttt{branch\_i}, \eventref{EV\_IFM\_REQUEST\_CANCEL\_01} shall be triggered.
        }

    \subsubsection{Fetch triggering}

      \event{EV\_IFM\_FETCH\_REQUEST\_01}{
          This event represents an instruction fetch request.
        }

      \req{D\_IFM\_FETCH\_TRIGGER\_01}{
          The instruction fetch module shall trigger \eventref{EV\_IFM\_FETCH\_REQUEST\_01} on the rising edge of \texttt{clk\_i} following the deassertion of \texttt{rst\_i}.
        }[
          derivedfrom=A\_INSTRUCTION\_FETCH\_01
        ]

      \req{D\_IFM\_FETCH\_TRIGGER\_02}{
          The instruction fetch module shall trigger \eventref{EV\_IFM\_FETCH\_REQUEST\_01} on the rising edge of \texttt{clk\_i} after both \texttt{output\_ready\_i} and \texttt{output\_valid\_o} are asserted.
        }[
          derivedfrom=A\_INSTRUCTION\_FETCH\_01
        ]

      \req{D\_IFM\_MEMORY\_FETCH\_01}{
          A 32-bit read cycle at the address represented by \texttt{pc} shall be initiated when \eventref{EV\_IFM\_FETCH\_REQUEST\_01} is triggered and no request cancellation is in progress.
        }

    \subsubsection{Wishbone interface}

      \begin{content}
          The following requirements are extracted from the Wishbone specification for implementing the memory interface of the instruction fetch module.
        \end{content}

      \req{D\_IFM\_WISHBONE\_DATASHEET\_01}{
          The memory interface shall comply with the Wishbone Datasheet provided in section \ref{user-needs}.
        }

      \req{D\_IFM\_WISHBONE\_RESET\_01}{
          The memory interface shall initialize itself at the rising edge of \texttt{clk\_i} following the assertion of \texttt{rst\_i}.
        }

      \req{D\_IFM\_WISHBONE\_RESET\_02}{
          The memory interface shall stay in the initialization state until the rising edge of \texttt{clk\_i} following the deassertion of \texttt{rst\_i}.
        }

      \req{D\_IFM\_WISHBONE\_RESET\_03}{
          Signals \texttt{wb\_stb\_o} and \texttt{wb\_cyc\_o} shall be deasserted while the memory interface is in the initialization state. The state of all other memory interface signals are undefined in response to a reset cycle.
        }

      \req{D\_IFM\_WISHBONE\_TRANSFER\_CYCLE\_01}{
          The memory interface shall assert \texttt{wb\_cyc\_o} for the entire duration of the memory access.
        }[
          rationale=TBC what wb\_cyc\_o does.
        ]

      \req{D\_IFM\_WISHBONE\_TRANSFER\_CYCLE\_02}{
          Signal \texttt{wb\_cyc\_o} shall be asserted no later than the rising edge of \texttt{clk\_i} that qualifies the assertion of \texttt{wb\_stb\_o}.
        }

      \req{D\_IFM\_WISHBONE\_TRANSFER\_CYCLE\_03}{
            Signal \texttt{wb\_cyc\_o} shall be deasserted no earlier than the rising edge of \texttt{clk\_i} that qualifies the deassertion of \texttt{wb\_stb\_o}.
        }

      \req{D\_IFM\_WISHBONE\_HANDSHAKE\_01}{
          The memory interface shall accept \texttt{wb\_ack\_i} signals at any time after a transaction is initiated.
        }

      \req{D\_IFM\_WISHBONE\_HANDSHAKE\_02}{
          The memory interface must qualify the following signals with \texttt{wb\_stb\_o} : \texttt{wb\_adr\_o}, \texttt{wb\_dat\_o}, \texttt{wb\_sel\_o} and \texttt{wb\_we\_o}.
        }

      \req{D\_IFM\_WISHBONE\_STALL\_01}{
          While initiating a request, the memory interface shall hold the state of its outputs until \texttt{wb\_stall\_i} is deasserted.
        }

      \vspace{0.5em}

      \begin{figure}[H]
          \centering
          \makeatletter\gdef\dividers{}
\begin{tikztimingtable}[%
    scale=0.7,
    timing/dslope=0.1,
    timing/.style={x=6ex,y=3ex},
    x=6ex,
    timing/rowdist=4ex,
    timing/name/.style={font=\footnotesize},
    timing/u/background/.style={fill=gray!20},
    timing/e/background/.style={fill=gray!20},
]
clk\_i & H 3{C C} L \\
adr\_o & 2U 2D{VALID} 4U \\
dat\_i & 2U 2U 2D{VALID} 2U \\
dat\_o & 8U \\
we\_o  & 8L \\
sel\_o & 2U 2D{VALID} 4U \\
stb\_o & 2L 2H 4L \\
ack\_i & 4L 2H 2L \\
cyc\_o & 2L 4H 2L \\
stall\_i & 8L \\
\extracode
% grid
\begin{pgfonlayer}{background}
\begin{scope}[semitransparent ,semithick]
\vertlines[darkgray,dotted]{2, 4, 6}
\dividers
\end{scope}
\end{pgfonlayer}
\end{tikztimingtable}

          \caption{Timing diagram of the single read cycle of the wishbone memory interface}
          \label{fig:ifm-wishbone-single-read-cycle}
        \end{figure}

      {
  \vspace{0.5em}
  \begin{center}
    \refstepcounter{table}
    Table \thetable: Description of the single read cycle of the wishbone memory interface defined in figure \ref{fig:ifm-wishbone-single-read-cycle}.\label{tab:ifm-wishbone-single-read-cycle}
  \end{center}

\footnotesize
\begin{xltabular}{0.9\textwidth}{|l|X|}
  \hline
  \cellcolor{gray!20}\textbf{CLOCK EDGE} & \cellcolor{gray!20}\textbf{DESCRIPTION} \\
  \hline
  \multirow{5}{*}{0} & The memory interface presents a valid address on \texttt{adr\_o} \\
  & The memory interface deasserts \texttt{we\_o} to indicate a READ cycle \\
  & The memory interface presents a bank select \texttt{sel\_o} to indicate where it expects data. \\
  & The memory interface asserts \texttt{cyc\_o} to indicate the start of the cycle. \\
  & The memory interface asserts stb\_o to indicate the start of the phase. \\
  \hline
  \multirow{3}{*}{1} & Valid data is provided on \texttt{dat\_i}. \\
  & \texttt{ack\_i} is asserted to indicate valid data. \\
  & The memory interface deasserts \texttt{stb\_o} to indicate end of data phase. \\
  \hline
  \multirow{3}{*}{2} & The memory interface latches data on \texttt{dat\_i}. \\
  & The memory interface deasserts \texttt{cyc\_o} to indicate the end of the cycle. \\
  & \texttt{ack\_i} is deasserted. \\
  \hline
\end{xltabular}
}


      \req{D\_IFM\_WISHBONE\_READ\_CYCLE\_01}{
          A read transaction shall be started by asserting both \texttt{wb\_cyc\_o} and \texttt{wb\_stb\_o}, and deasserting \texttt{wb\_we\_o}.
        }

      \req{D\_IFM\_WISHBONE\_READ\_CYCLE\_02}{
          The memory interface shall conform to the READ cycle detailed in figure \ref{fig:wishbone-single-read-cycle}.
        }


      \vspace{0.5em}

      \begin{content}
          The memory write cycles are not implemented in this module as it shall only read data from memory.
        \end{content}

      \req{D\_IFM\_WISHBONE\_TIMING\_01}{
          The clock input \texttt{clk\_i} shall coordinate all activites for the internal logic within the memory interface. All output signals of the memory interface shall be registered at the rising edge of \texttt{clk\_i}. All input signals of the memory interface shall be stable before the rising edge of \texttt{clk\_i}.
        }[
          rationale={As long as the memory interface is designed within the clock domain of \texttt{clk\_i}, the requirement will be satisfied by using the place and route tool.}
        ]


      \begin{content}
          BLOCK cycles are not supported in revision 1.0.0.
        \end{content}

    \subsubsection{Output}

      \req{D\_IFM\_OUTPUT\_01}{
          The signal \texttt{instr\_o} shall be set to the value represented by \texttt{wb\_dat\_i} on the rising edge of \texttt{clk\_i} following the assertion of \texttt{wb\_ack\_i}
        }

      \req{D\_IFM\_OUTPUT\_02}{
          The signal \texttt{pc\_o} shall be set to the value of the \texttt{pc} register on the rising edge of \texttt{clk\_i} following the assertion of \texttt{wb\_ack\_i}
        }

      \req{D\_IFM\_OUTPUT\_HANDSHAKE\_01}{
          The signal \texttt{output\_valid\_o} shall be deasserted on the rising edge of \texttt{clk\_i} when \eventref{EV\_IFM\_FETCH\_REQUEST\_01} is triggered.
        }[
          rationale=Refer to section \ref{control-hazard}.
        ]

      \req{D\_IFM\_OUTPUT\_HANDSHAKE\_02}{
          The signal \texttt{output\_valid\_o} shall be asserted on the rising edge of \texttt{clk\_i} following the assertion of \texttt{wb\_ack\_i}.
        }[
          rationale=Refer to section \ref{control-hazard}.
        ]

      \req{D\_IFM\_OUTPUT\_HANDSHAKE\_03}{
          When \texttt{output\_valid\_o} is asserted, the instruction fetch module shall hold the value of the \texttt{instr\_o} signal until the rising edge of \texttt{clk\_i} following the assertion of \texttt{output\_ready\_i}.
        }[
          rationale=Refer to section \ref{pipeline-stall}.
        ]

      \req{D\_IFM\_OUTPUT\_HANDSHAKE\_04}{
          When \texttt{output\_valid\_o} is asserted, the instruction fetch module shall hold the value of the \texttt{pc\_o} signal until the rising edge of \texttt{clk\_i} following the assertion of \texttt{output\_ready\_i}.
        }[
          rationale=Refer to section \ref{pipeline-stall}.
        ]

  \subsection{Behavior}

    \begin{content}
        This module doesn't contain any prefetch mechanism as there is no performance requirement for version 1.0.0. This will lead to a performance bottleneck due to the number of cycles needed for fetching instructions from memory.
      \end{content}

    \subsubsection{Normal behavior}

      \begin{content}
          During normal operation, the instruction fetch module sequentially emits memory requests to address stored in the \texttt{pc} register. Upon successful request, \texttt{pc} is incremented to point to the following instruction.

          This behavior induces a pipeline stall due to the delay of the wishbone interface.
        \end{content}

      \begin{figure}[H]
          \centering
          \makeatletter\gdef\dividers{}
\begin{tikztimingtable}[%
    scale=0.7,
    timing/dslope=0.1,
    timing/.style={x=5ex,y=3ex},
    x=5ex,
    timing/rowdist=4ex,
    timing/name/.style={font=\footnotesize},
    timing/u/background/.style={fill=gray!20},
    timing/e/background/.style={fill=gray!20},
]
clk\_i & H 5{C C} L \\
rst\_i & 2E 8L 2E\\
& \divider{Memory access} \\
wb\_clk\_i & H 5{C C} L \\
  wb\_adr\_o[31:0] & 2.5U 4D{pc} 4D{pc + 4} 1.5U \\
  wb\_dat\_i[31:0] & 4U 2D{mem[pc]} 2U 2D{mem[pc + 4]} 2U \\
wb\_stb\_o & 2.5E 8H 1.5E \\
wb\_ack\_i & 2E 2L 2H 2L 2H 2E \\
& \divider{Stage outputs} \\
output\_ready\_i & 2E 8H 2E \\
output\_valid\_o & 2.5E 2L 2H 2L 2H 1.5E\\
instr\_o[31:0] & 4.5U 2D{mem[pc]} 2U 2D{mem[pc + 4]} 1.5U \\
& \divider{Internal pc value} \\
pc & 2U 4D{pc} 4D{pc + 4} 2U \\
\extracode
% grid
\begin{pgfonlayer}{background}
\begin{scope}[semitransparent ,semithick]
\vertlines[darkgray,dotted]{2, 4, 6, 8, 10}
\dividers
\end{scope}
\end{pgfonlayer}
\end{tikztimingtable}

          \caption{Timing diagram of the normal behavior of the instruction fetch module}
          \label{fig:ifm-behavior-normal}
        \end{figure}

    \subsubsection{Reset behavior}

      \begin{content}
          Once a reset occurs, \texttt{pc} is loaded with the boot address before returning to normal operation.

          This behavior doesn't induce any pipeline stall \textit{per se} as the pipeline is reset during this operation.
        \end{content}

      \begin{figure}[H]
          \centering
          \makeatletter\gdef\dividers{}
\begin{tikztimingtable}[%
    scale=0.7,
    timing/dslope=0.1,
    timing/.style={x=5ex,y=3ex},
    x=5ex,
    timing/rowdist=4ex,
    timing/name/.style={font=\footnotesize},
    timing/u/background/.style={fill=gray!20},
    timing/e/background/.style={fill=gray!20},
]
clk\_i & H 7{C C} L \\
rst\_i & 2E 6L 2H 4L 2E\\
& \divider{Memory access} \\
wb\_clk\_i & H 7{C C} L \\
wb\_rst\_o & 2U 6L 2H 4L 2E \\
  wb\_adr\_o[31:0] & 2.5U 4D{pc} 2D{pc + 4} 2U 4D{boot\_addr} 1.5U \\
  wb\_dat\_i[31:0] & 4U 2D{mem[pc]} 2U 2U 2U 2D{\textit{x}} 2U \\
wb\_stb\_o & 2.5E 6H 2L 4H 1.5E \\
wb\_ack\_i & 2E 2L 2H 2L 2L 2L 2H 2E \\
& \divider{Stage outputs} \\
ready\_i & 2E 12H 2E \\
valid\_o & 2.5E 2L 2H 2L 2L 2L 2H 1.5E\\
instr\_o[31:0] & 4.5U 2D{mem[pc]} 2U 2U 2U 2D{\textit{x}} 1.5U \\
& \divider{Internal pc value} \\
pc & 2U 4D{pc} 2D{pc + 4} 6D{boot\_addr} 2U \\
\extracode
% grid
\begin{pgfonlayer}{background}
\begin{scope}[semitransparent ,semithick]
\vertlines[darkgray,dotted]{2, 4, 6, 8, 10, 12, 14}
\dividers
\end{scope}
\end{pgfonlayer}
\end{tikztimingtable}
\begin{center}
  \scriptsize \textit{x : mem[boot\_addr]}
\end{center}

          \caption{Timing diagram of the reset behavior of the instruction fetch module}
          \label{fig:ifm-behavior-reset}
        \end{figure}

    \subsubsection{Resource busy behavior}

      \begin{content}
          The instruction fetch module is capable of handling wait states from memory through the stalling mechanism.
        \end{content}

      \begin{figure}[H]
          \centering
          \makeatletter\gdef\dividers{}
\begin{tikztimingtable}[%
    scale=0.7,
    timing/dslope=0.1,
    timing/.style={x=5ex,y=3ex},
    x=5ex,
    timing/rowdist=4ex,
    timing/name/.style={font=\footnotesize},
    timing/u/background/.style={fill=gray!20},
    timing/e/background/.style={fill=gray!20},
]
clk\_i & H 7{C C} L \\
rst\_i & 2E 12L 2E\\
& \divider{Memory access} \\
wb\_clk\_i & H 7{C C} L \\
wb\_adr\_o[31:0] & 2.5U 4D{pc} 8D{pc + 4} 1.5U \\
wb\_dat\_i[31:0] & 4U 2D{mem[pc]} 2U 2U 2U 2D{mem[pc + 4]} 2U \\
wb\_stb\_o & 2.5E 12H 1.5E \\
wb\_ack\_i & 2E 2L 2H 2L 2L 2L 2H 2E \\
& \divider{Stage outputs} \\
ready\_i & 2E 12H 2E \\
valid\_o & 2.5E 2L 2H 2L 2L 2L 2H 1.5E\\
instr\_o[31:0] & 4.5U 2D{mem[pc]} 2U 2U 2U 2D{mem[pc + 4]} 1.5U \\
& \divider{Internal pc value} \\
pc & 2U 4D{pc} 8D{pc + 4} 2U \\
\extracode
% grid
\begin{pgfonlayer}{background}
\begin{scope}[semitransparent ,semithick]
\vertlines[darkgray,dotted]{2, 4, 6, 8, 10, 12, 14}
\dividers
\end{scope}
\end{pgfonlayer}
\end{tikztimingtable}

          \caption{Timing diagram of the memory resource busy behavior of the instruction fetch module}
          \label{fig:ifm-behavior-wait}
        \end{figure}

    \subsubsection{Jump behavior}

      \begin{content}
          TBC
        \end{content}

      \begin{figure}[H]
          \centering
          \makeatletter\gdef\dividers{}
\begin{tikztimingtable}[%
    scale=0.7,
    timing/dslope=0.1,
    timing/.style={x=5ex,y=3ex},
    x=5ex,
    timing/rowdist=4ex,
    timing/name/.style={font=\footnotesize},
    timing/u/background/.style={fill=gray!20},
    timing/e/background/.style={fill=gray!20},
]
clk\_i & H 7{C C} L \\
rst\_i & 2E 12L 2E\\
& \divider{Jump inputs} \\
branch\_i & \\
boffset\_i[19:0] & \\
& \divider{Memory access} \\
wb\_clk\_i & H 7{C C} L \\
  wb\_adr\_o[31:0] & 2.5U 4D{pc} 4D{pc + 4} 4D{pc + 8} 1.5U \\
  wb\_dat\_i[31:0] & 4U 2D{mem[pc]} 2U 2D{mem[pc + 4]} 2U 2D{mem[pc + 8]} 2U \\
wb\_stb\_o & 2.5E 12H 1.5E \\
wb\_ack\_i & 2E 2L 2H 2L 2H 2L 2H 2E \\
& \divider{Stage outputs} \\
output\_ready\_i & 2E 12H 2E \\
output\_valid\_o & 2.5E 2L 2H 2L 2H 2L 2H 1.5E\\
instr\_o[31:0] & 4.5U 2D{mem[pc]} 2U 2D{mem[pc + 4]} 2U 2D{mem[pc + 8]} 1.5U \\
\extracode
% grid
\begin{pgfonlayer}{background}
\begin{scope}[semitransparent ,semithick]
\vertlines[darkgray,dotted]{2, 4, 6, 8, 10, 12, 14}
\dividers
\end{scope}
\end{pgfonlayer}
\end{tikztimingtable}

          \caption{Timing diagram of the jump behavior of the instruction fetch module for branch events}
          \label{fig:ifm-behavior-branch}
        \end{figure}

      \begin{figure}[H]
          \centering
          \input{arch/figures/8_ifm-behavior-jump-interrupt.tex}
          \caption{Timing diagram of the jump behavior of the instruction fetch module for interrupt events}
          \label{fig:ifm-behavior-interrupt}
        \end{figure}

    \subsubsection{Hazard behaviors}

      \begin{content}
          Hazard behaviors are described in section \ref{pipeline-stall}.
        \end{content}

\newpage
