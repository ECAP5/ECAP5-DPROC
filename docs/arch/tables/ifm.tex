{
  \vspace{0.5em}
  \begin{center}
    \refstepcounter{table}
    Table \thetable: Instruction Fetch Module interface signals\label{tab:ifm-interface}
  \end{center}

\footnotesize
\begin{xltabular}{0.9\textwidth}{|l|c|c|X|}
  \hline
  \cellcolor{gray!20}\textbf{NAME} & \cellcolor{gray!20}\textbf{TYPE} & \cellcolor{gray!20}\textbf{WIDTH} & \cellcolor{gray!20}\textbf{DESCRIPTION} \\
  \hline
  clk\_i & I & 1 & Clock input. \\
  \hline
  rst\_i & I & 1 & Reset input. \\
  \hline
  \multicolumn{4}{|l|}{\textbf{JUMP LOGIC}} \\
  \hline
  irq\_i & I & 1 & External interrupt request. \\
  \hline
  drq\_i & I & 1 & External debug request. \\
  \hline
  branch\_i & I & 1 & Branch request. \\
  \hline
  boffset\_i & I & TBC & Branch offset. TBC \\
  \hline
  \multicolumn{4}{|l|}{\textbf{WISHBONE MASTER}} \\
  \hline
  wb\_clk\_o & O & 1 & Wishbone clock output. This is hardwired to clk\_i. \\
  \hline
  wb\_adr\_o & O & 32 & Wishbone read address.  \\
  \hline
  wb\_dat\_i & I & 32 & Wishbone read data. \\
  \hline
  wb\_stb\_o & O & 1 & Strobe output indicates a valid data transfer cycle. \\
  \hline
  wb\_ack\_i & I & 1 & Acknowledge. Indicates a normal termination of a bus cycle. \\
  \hline
  wb\_cyc\_o & O & 1 & Cycle. Indicates that a valid bus cycle is in progress. \\
  \hline
  \multicolumn{4}{|l|}{\textbf{OUTPUT LOGIC}} \\
  \hline
  ready\_i & I & 1 & Asserted when the output is ready to be received. \\
  \hline
  valid\_o & O & 1 & Asserted when the output is ready to be sent. \\
  \hline
  instr\_o & O & 32 & Instruction to be executed. \\
  \hline
\end{xltabular}
}
