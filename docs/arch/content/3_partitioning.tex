\section{Functional Partitioning}

\begin{content}
  ECAP5-DPROC is built around a pipelined architecture with the following stages :
  \begin{itemize}
    \vspace{-0.5em}
    \item The instruction fetch stage loads the next instruction from memory.
    \vspace{-0.5em}
    \item The decode stage handles the instruction decoding to provide the next stage with the different instruction input values including reading from internal registers.
    \vspace{-0.5em}
    \item The execute stage implements instruction behaviors. This includes performing integer operations as well as accessing memory.
    \vspace{-0.5em}
    \item The write-back stage which handles storing instructions outputs to internal registers.
  \end{itemize}

\begin{figure}[h!]
    \centering
    \vspace{1em}
\scalebox{0.75}{
\begin{tikzpicture}[scale=1.25, draw=gray, inner sep=0, outer sep=0]
  \node[rectangle, draw=black,
    minimum height = 1.25cm,
    minimum width = 10cm,
    fill = gray!20] (EMM) at (6, -2.5) {EMM};
  \node[rectangle, draw=black,
    minimum height = 2cm,
    minimum width = 2cm,
    fill = blue!20!gray!20] (IFM) at (3, 0) {IFM};
  \node[rectangle, draw=black,
    minimum height = 2cm,
    minimum width = 2cm,
    fill = blue!20!gray!20] (DECM) at (6, 0) {DECM};
  \node[rectangle, draw=black,
    minimum height = 1.25cm,
    minimum width = 10cm,
    fill = gray!20] (REGM) at (9, 2.5) {REGM};
  \node[rectangle, draw=black,
    minimum height = 2cm,
    minimum width = 2cm,
    fill = blue!20!gray!20] (EXM) at (9, 0) {EXM};
  \node[rectangle, draw=black,
    minimum height = 2cm,
    minimum width = 2cm,
    fill = blue!20!gray!20] (WBM) at (12, 0) {WBM};

  \draw[->] (IFM.east) -- (DECM.west);
  \draw[->] (DECM.east) -- (EXM.west);
  \draw[->] (EXM.east) -- (WBM.west);

  \draw[<-] (IFM.south) -- (IFM.south|- EMM.north);

  \draw[->] ([xshift=-0.25cm]EXM.south) -- ([xshift=-0.25cm]EXM.south |- EMM.north);
  \draw[<-] ([xshift=0.25cm]EXM.south) -- ([xshift=0.25cm]EXM.south |- EMM.north);

  \draw[<-] (DECM.north) -- (DECM.north |- REGM.south);
  \draw[->] (WBM.north) -- (WBM.north |- REGM.south);

  % surrounding rectangle
  \node[dashed, draw=black, align=center, inner sep=0.5cm, fit=(EMM) (IFM) (DECM) (EXM) (WBM) (REGM)] (border) {};
  \node[anchor=north west, inner sep=0.5cm] (border-text) at (border.north west) {TOPM};

  % external interface
  \node (irq) at ([yshift=0.25cm]IFM.west) {};
  \node (drq) at ([yshift=-0.25cm]IFM.west) {};
  \node (axi) at (EMM.west) {};
  \node (clk) at ([yshift=-1.25cm]border.north west) {};
  \node (rst) at ([yshift=-1.75cm]border.north west) {};

  \node (extend) at ([xshift=-1.5cm]irq.center) {};

  \draw[->] (clk.center -| extend.center) node[left=0.2cm, anchor=east]{\small CLK} -- (clk.center);
  \draw[->] (rst.center -| extend.center) node[left=0.2cm, anchor=east]{\small RST\_N} -- (rst.center);
  \draw[->] (irq.center -| extend.center) node[left=0.2cm, anchor=east]{\small IRQ} -- (irq.center);
  \draw[->] (drq.center -| extend.center) node[left=0.2cm, anchor=east]{\small DRQ} -- (drq.center);
  \draw[<->, thick] (axi.center -| extend.center) node[left=0.2cm, anchor=east]{\small AXI-Lite} -- (axi.center);

  % clk triangle
  \draw[-, dashed] ([yshift=0.25cm]clk.center) -- ([xshift=0.25cm]clk.center) -- ([yshift=-0.25cm]clk.center);

\end{tikzpicture}
}

    \caption{Schematic view of the architecture of ECAP5-DPROC}
    \label{fig:architecture}
\end{figure}

  The design is split into the following functional modules :
  \begin{itemize}
    \vspace{-0.5em}
    \item The \textbf{Top Module} (TOPM) which integrates all other modules.
    \item The \textbf{External Memory Module} (EMM), in charge of accessing memory and peripherals.
    \vspace{-0.5em}
    \item The \textbf{Instruction Fetch Module} (IFM), in charge of implementing the instruction fetch stage.
    \vspace{-0.5em}
    \item The \textbf{Decode Module} (DECM), in charge of implementing the decode stage.
    \vspace{-0.5em}
    \item The \textbf{Register Module} (REGM), implementing the internal registers.
    \vspace{-0.5em}
    \item The \textbf{Execute Module} (EXM), in charge of implementing the execute stage.
    \item The \textbf{Write-Back Module} (WBM), in charge of implementing the write-back stage.
  \end{itemize}
\end{content}

\newpage

\section{Top Module}

\begin{content}
  Handshaking and bubbling
\end{content}

\newpage

\section{External Memory Module}
\newpage

\section{Instruction Fetch Module}
\begin{figure}[h!]
    \centering
    \vspace{1em}
\scalebox{0.85}{
\begin{tikzpicture}[scale=1.25, draw=gray, inner sep=0, outer sep=0]
  \node[rectangle, draw=white,
    minimum height = 5cm,
    minimum width = 2cm] (back) at (0, 0) {};
  \node[rectangle, draw=black,
    anchor=south,
    align=center,
    minimum height = 4.5cm,
    minimum width = 2.5cm,
    fill = gray!20] (mem) at (back.south) {Memory\\read};
  \node[rectangle, draw=black,
    align=center,
    anchor=north east,
    minimum height = 0.75cm,
    minimum width = 2cm,
    fill = blue!20!gray!20] (pc) at ([xshift=-1cm]mem.north west) {pc};
  \node[rectangle, draw=black,
    align=center,
    anchor=south east,
    minimum height = 3cm,
    minimum width = 2cm,
    fill = gray!20] (jmp) at ([xshift=-1cm]mem.south west) {Jump \\ logic};

  \node[dashed, draw=black, align=center, inner sep=1cm, fit=(jmp) (pc) (mem) (back)] (border) {};
  \node[anchor=north west, inner sep=0.25cm] (border-text) at (border.north west) {IFM};

  \node (cross-pc-mem) at (pc.east -| mem.west) {};
  \draw[->] (pc.east) -- (cross-pc-mem.center);

  \draw[->] (jmp.north) -- (pc.south);

  \node (rport1) at (mem.east) {};
  \draw[<-] ([xshift=1.5cm]rport1.center) node[right=0.2cm, anchor=west]{\footnotesize instr\_o[31:0]} -- (rport1.center);

  \node(uport1) at (mem.north) {};
  \draw[<-] ([yshift=2cm]uport1.center) node[above=0.2cm, anchor=south]{\footnotesize Wishbone} -- (uport1.center);

  \node (lport2) at ([yshift=0.4cm]jmp.west) {};
  \node (lport1) at ([yshift=0.4cm]lport2.center) {};
  \node (lport3) at ([yshift=-0.4cm]jmp.west) {};
  \node (lport4) at ([yshift=-0.4cm]lport3.center) {};

  \draw[->] ([xshift=-1.5cm]lport1.center) node[left=0.2cm, anchor=east]{\footnotesize irq\_i} -- (lport1.center);
  \draw[->] ([xshift=-1.5cm]lport2.center) node[left=0.2cm, anchor=east]{\footnotesize drq\_i} -- (lport2.center);
  \draw[->] ([xshift=-1.5cm]lport3.center) node[left=0.2cm, anchor=east]{\footnotesize branch\_i} -- (lport3.center);
  \draw[->] ([xshift=-1.5cm]lport4.center) node[left=0.2cm, anchor=east]{\footnotesize boffset\_i} -- (lport4.center);

  \node (clk) at ([yshift=-0.9cm]border.north west) {};
  \draw[->] ([xshift=-1.5cm]lport3.center |- clk.center) node[left=0.2cm, anchor=east]{\small clk\_i} -- (clk.center);
  % clk triangle
  \draw[-, dashed, draw=black] ([yshift=0.25cm]clk.center) -- ([xshift=0.25cm]clk.center) -- ([yshift=-0.25cm]clk.center);

  \node (rst) at (pc.west) {};
  \draw[->] ([xshift=-1.5cm]rst.center) node[left=0.2cm, anchor=east]{\small rst\_i} -- (rst.center);
\end{tikzpicture}
}

    \caption{Schematic view of the Instruction Fetch Module}
    \label{fig:regm}
\end{figure}

\begin{content}
The instruction fetch module handles fetching from memory the instructions to be executing. The signals are described in table \ref{tab:ifm-interface}. 
\end{content}

\begin{table}[H]
  \centering
  {
  \vspace{0.5em}
  \begin{center}
    \refstepcounter{table}
    Table \thetable: Instruction Fetch Module interface signals\label{tab:ifm-interface}
  \end{center}

\footnotesize
\begin{xltabular}{0.9\textwidth}{|l|c|c|X|}
  \hline
  \cellcolor{gray!20}\textbf{NAME} & \cellcolor{gray!20}\textbf{TYPE} & \cellcolor{gray!20}\textbf{WIDTH} & \cellcolor{gray!20}\textbf{DESCRIPTION} \\
  \hline
  clk\_i & I & 1 & Clock input. \\
  \hline
  rst\_i & I & 1 & Reset input. \\
  \hline
  \multicolumn{4}{|l|}{\textbf{JUMP LOGIC}} \\
  \hline
  irq\_i & I & 1 & External interrupt request. \\
  \hline
  drq\_i & I & 1 & External debug request. \\
  \hline
  branch\_i & I & 1 & Branch request. \\
  \hline
  boffset\_i & I & TBC & Branch offset. TBC \\
  \hline
  \multicolumn{4}{|l|}{\textbf{WISHBONE MASTER}} \\
  \hline
  wb\_clk\_o & O & 1 & Wishbone clock output. This is hardwired to clk\_i. \\
  \hline
  wb\_adr\_o & O & 32 & Wishbone read address.  \\
  \hline
  wb\_dat\_i & I & 32 & Wishbone read data. \\
  \hline
  wb\_stb\_o & O & 1 & Strobe output indicates a valid data transfer cycle. \\
  \hline
  wb\_ack\_i & I & 1 & Acknowledge. Indicates a normal termination of a bus cycle. \\
  \hline
  wb\_cyc\_o & O & 1 & Cycle. Indicates that a valid bus cycle is in progress. \\
  \hline
  \multicolumn{4}{|l|}{\textbf{OUTPUT LOGIC}} \\
  \hline
  ready\_i & I & 1 & Asserted when the output is ready to be received. \\
  \hline
  valid\_o & O & 1 & Asserted when the output is ready to be sent. \\
  \hline
  instr\_o & O & 32 & Instruction to be executed. \\
  \hline
\end{xltabular}
}

  \caption{Instruction Fetch Module interface signals}
  \label{tab:ifm-interface}
\end{table}

\subsection{Jump logic}

\subsection{Wishbone master}

\subsection{Output logic}

\newpage
\section{Decode Module}
\newpage

\section{Register Module}

\begin{figure}[h!]
    \centering
    \vspace{1em}
\scalebox{0.85}{
\begin{tikzpicture}[scale=1.25, draw=gray, inner sep=0, outer sep=0]
  \node[rectangle, draw=white,
    align=center,
    minimum height = 4.75cm,
    minimum width = 4cm,
    fill = white] (back) at (6, 2.5) {Block RAM \\ \small DP16KD};
  \node[rectangle, draw=black,
    align=center,
    anchor=south,
    minimum height = 4cm,
    minimum width = 4cm,
    fill = blue!30!gray!20] (BRAM) at (back.south) {Block RAM \\ \small DP16KD};

  \node[rectangle, draw=black,
    align=center,
    anchor=east,
    minimum height = 4cm,
    minimum width = 1.5cm,
    fill = gray!20] (glue1) at ([xshift=-0.5cm]BRAM.west) {};
  \node[rotate=90] (glue1text) at (glue1.center) {Read logic};

  \node[rectangle, draw=black,
    align=center,
    anchor=west,
    minimum height = 4cm,
    minimum width = 1.5cm,
    fill = gray!20] (glue2) at ([xshift=0.5cm]BRAM.east) {};
  \node[rotate=90] (glue2text) at (glue2.center) {Write logic};

  \node[dashed, draw=black, align=center, inner sep=1cm, fit=(BRAM) (back) (glue1) (glue2)] (border) {};
  \node[anchor=north west, inner sep=0.25cm] (border-text) at (border.north west) {REGM};

  \node (lport2) at ([yshift=0.5cm]glue1.west) {};
  \node (lport1) at ([yshift=0.5cm]lport2.center) {};
  \draw[->] ([xshift=-1.5cm]lport1.center) node[left=0.2cm, anchor=east]{\small raddr1\_i[4:0]} -- (lport1.center);
  \draw[<-] ([xshift=-1.5cm]lport2.center) node[left=0.2cm, anchor=east]{\small rdata1\_o[31:0]} -- (lport2.center);

  \node (lport3) at ([yshift=-0.5cm]glue1.west) {};
  \node (lport4) at ([yshift=-0.5cm]lport3.west) {};
  \draw[->] ([xshift=-1.5cm]lport3.center) node[left=0.2cm, anchor=east]{\small raddr2\_i[4:0]} -- (lport3.center);
  \draw[<-] ([xshift=-1.5cm]lport4.center) node[left=0.2cm, anchor=east]{\small rdata2\_o[31:0]} -- (lport4.center);

  \node (rport2) at (glue2.east) {};
  \node (rport1) at ([yshift=0.5cm]rport2.center) {};
  \node (rport3) at ([yshift=-0.5cm]rport2.center) {};
  \draw[->] ([xshift=1.5cm]rport1.center) node[right=0.2cm, anchor=west]{\small waddr\_i[4:0]} -- (rport1.center);
  \draw[->] ([xshift=1.5cm]rport2.center) node[right=0.2cm, anchor=west]{\small write\_i} -- (rport2.center);
  \draw[->] ([xshift=1.5cm]rport3.center) node[right=0.2cm, anchor=west]{\small wdata\_i[31:0]} -- (rport3.center);

  \node (clk) at ([yshift=-1cm]border.north west) {};
  \draw[->] ([xshift=-1.5cm]lport3.center |- clk.center) node[left=0.2cm, anchor=east]{\small clk\_i} -- (clk.center);
  % clk triangle
  \draw[-, dashed, draw=black] ([yshift=0.25cm]clk.center) -- ([xshift=0.25cm]clk.center) -- ([yshift=-0.25cm]clk.center);

  \draw[<->] (glue1.east) -- (BRAM.west);
  \draw[<->] (glue2.west) -- (BRAM.east);
\end{tikzpicture}
}

    \caption{Schematic view of the Register Module}
    \label{fig:regm}
\end{figure}

\begin{content}
The register module implements the 32 internal registers of ECAP5-DPROC. It has two reading port and one writing port. The signals are described in table \ref{tab:regm-interface}. 
\end{content}

\begin{table}[H]
  \centering
  {
\footnotesize
\begin{tabularx}{0.9\textwidth}{|l|c|c|X|}
  \hline
  \cellcolor{gray!20}\textbf{NAME} & \cellcolor{gray!20}\textbf{TYPE} & \cellcolor{gray!20}\textbf{WIDTH} & \cellcolor{gray!20}\textbf{DESCRIPTION} \\
  \hline
  clk\_i & I & 1 & Clock input. \\
  \hline
  \multicolumn{4}{|l|}{\textbf{FIRST READING PORT}} \\
  \hline
  raddr1\_i & I & 5 & Register selector. \\
  \hline
  rdata1\_o & O & 32 & Selected register value. \\
  \hline
  \multicolumn{4}{|l|}{\textbf{SECOND READING PORT}} \\
  \hline
  raddr2\_i & I & 5 & Register selector. \\
  \hline
  rdata2\_o & O & 32 & Selected register value. \\
  \hline
  \multicolumn{4}{|l|}{\textbf{WRITING PORT}} \\
  \hline
  waddr\_i & I & 5 & Register selector. \\
  \hline
  write\_i & I & 1 & Asserted to indicate a write. \\
  \hline
  wdata\_i & I & 32 & Data to be written. \\
  \hline
\end{tabularx}
}

  \caption{Register Module interface signals}
  \label{tab:regm-interface}
\end{table}

\begin{content}
When reading, rdata1\_i and rdata2\_i output, on the rising edge of clk\_i, the value of the register respectively selected by raddr1\_i and raddr2\_i. A register write happens on the rising edge of clk\_i when write\_i is asserted, writing the value wdata\_i in the register selected by waddr\_i.
\end{content}

\newpage

\section{Execute Module}
\newpage

\section{Write-Back Module}
\newpage

\section{Debug}
\newpage
